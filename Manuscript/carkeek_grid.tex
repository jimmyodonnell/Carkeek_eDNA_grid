%--------------------------------------||--------------------------------------%

\documentclass[11pt,letterpaper]{article} %  font size


%-----------------------------------PACKAGES-----------------------------------%

\usepackage[T1]{fontenc} % Choose an output font encoding (T1) that has support for the accented characters used by the most widespread European languages
\usepackage[utf8]{inputenc} % Allow input of accented characters (and more...)
\usepackage{graphicx} %  figures
\usepackage[round]{natbib} % names in citations
\usepackage{lineno} % line numbers
\usepackage{authblk} % allows more intuitive formatting for multiple authors/affiliations
\usepackage[margin=1in]{geometry} % make the margins 1 inch on all sides of the document
%\usepackage{amsmath} % useful for formatting math stuff, especially complex equations
%\usepackage{pdflscape} % rotate table into landscape mode
%\usepackage{subfigure} % side by side figures
%\usepackage{longtable} % for tables that span multiple pages.
\usepackage{setspace} % double spacing



%----------------------------------FORMATTING-----------------------------------

%\topmargin -1cm %0.0cm
%\textwidth 16cm % what does this do?
%\textheight 21cm % what does this do?
%\footskip 1.0cm % what does this do?
%\oddsidemargin 0.0cm

\date{\today}
\doublespacing % initiate double spacing (package setspace)
%\linespread{2} % alternate method of double spacing


%------------------------------------TITLE--------------------------------------

\title{There once was a grid at ol' Carkeek}

%-----------------------------------AUTHORS-------------------------------------

% using package 'authblk':
\author[1]{First Author\thanks{first.author@funstuff.com}}

\author[1,2]{Second Author}

\author[2]{Third Author}

\affil[1]{Department of Computer Science, \LaTeX\ University}
\affil[2]{Department of Mechanical Engineering, Superfabulous University}



%----------------------------------FORMATTING-----------------------------------
\begin{document}
\maketitle
\linenumbers % start line numbers
\def\linenumberfont{\normalfont\small\rmfamily} % change line number font


%----------------------------------KEYWORDS-------------------------------------
\section*{Keywords}
Stuff, things, neat, cool, wow, instafun, tags4likes, etc

%----------------------------------ABSTRACT-------------------------------------
\section*{Abstract}
This is the text of the abstract.

%---------------------------------INTRODUCTION----------------------------------
\section*{Introduction}
For centuries, humankind has wondered: If I have two apples, and someone gives me another two apples, how many apples do I have? Some people did this \citep{Darwin1859}.

%-----------------------------------METHODS-------------------------------------
\section*{Methods}
We use the general framework outlined by Shelton et al (CITE).
That study outlined the structure for estimation of the proportional biomass of a taxon ($B_i$) given the proportional counts of sequences recovered from a parallel sequencing run ($Z_i$).

We modeled the counts of DNA sequences ($Z$) from each of a given taxon $i$, in each replicate PCR $j$, from each replicate of a given location $k$ (hence, $Z_{ijk}$), as though they are ?(proportional to/drawn from)? a Poisson distribution. A Poisson distribution is described by one and only one parameter, $\lambda$, which is equal to both the mean and variance. Because in this case our modeled values are discrete counts, we use the natural exponent, $e^\lambda$. % WHY IS IT EXPONENTIATED? COUNTS?
Thus,
\begin{equation}\label{some_cool_eqn_name}
	Z_{ijk} \sim Poisson(e^{\lambda_{ijk}})
\end{equation}

In turn, we further assume this parameter $\lambda$ is linearly proportional to a suite of taxon-, pcr-, and site- specific parameters describing the variance associated with each sub-process linking the amount of DNA ($Y$) of a given taxon $i$ at a given location $k$ in a DNA extract (hence $Y_{ik}$):

\begin{equation}\label{GLM}
	\lambda_{ijk} = \beta_0 + \beta_i + \eta_{ijk} + \epsilon_{ijk}
\end{equation}

Where $\beta_0$ is a general intercept across all taxa, $\beta_i$ is a fixed effect accounting for the variance associated with taxon $i$, and $\eta_{ijk}$ and $\epsilon_{ijk}$ are random effects of variance resulting from the processes associated with PCR and spatial location, respectively.

%We constructed the following mathematical model (\ref{my_equation_label}) to better understand the concept:
% using '\ref{something}' in the text will refer to any object (e.g. figure, equation, table) which contains the corresponding '\label{something}'
% Note that you need to run latex a few times to get it to register numbers correctly

%\begin{equation}\label{my_equation_label}
%	Y = 2a + 2a
%\end{equation}
%
%Where $a$ represents an apple.

%-----------------------------------RESULTS-------------------------------------
\section*{Results}
We found that if you have two apples, and someone gives you another two apples, you have four apples.

%----------------------------------DISCUSSION-----------------------------------
\section*{Discussion}
Boy those results sure are neat. Now, the pressing question becomes: How do you like them apples?

%-------------------------------ACKNOWLEDGEMENTS--------------------------------
\section*{Acknowledgements}
We wish to thank all of the little people.

%-----------------------------------FUNDING-------------------------------------
\section*{Funding}
This study was funded by our super-rich uncle.

%--------------------------------CONTRIBUTIONS----------------------------------
\section*{Author Contributions}
Conceived and designed the experiments: James L. O'Donnell, Ryan P. Kelly, A. Ole Shelton.
Collected the data: James L. O'Donnell, Greg Williams, Natalie C. Lowell, Ryan P. Kelly, A. Ole Shelton, .
Conducted the analyses: .
Wrote the first draft: .
Edited the manuscript: .


%-------------------------------------DATA-------------------------------------%
\section*{Data Availablity}
% Are the data and code available in a permanent, publicly accessible data archive or repository?
The data and code used to generate our results can be found at the following url: 


%----------------------------------REFERENCES-----------------------------------
%\section*{References} % commented out because the section title is automatically inserted if using an automatically-generated bibliography

\bibliographystyle{apalike} % or: plain,unsrt,alpha,abbrv,acm,apalike,ieeetr
\bibliography{/Users/threeprime/Documents/Publications/bibtex/library} % path to your .bib file excluding .bib extension (e.g. /Users/threeprime/Documents/Publications/bibtex/library)


%-----------------------------------FIGURES-------------------------------------
\section*{Figures}

%\begin{figure}[h!] % [h!] forces the figure to be placed roughly here
%  \centering
%    \includegraphics[width=1\textwidth]{figure_filename.pdf}
%    \caption{This is the figure caption.}
%  \label{myfigure} % use this to refer to your figure in the text, so that numbering updates automatically
%\end{figure}


%----------------------------------SUPPLEMENT-----------------------------------
%\section*{Supplemental Material}




\end{document}